\chapter{Conclusion} %synthesis not summary..distill key and best ideas: 

%intro
In light of the science objectives of mm-VLBI observations and software advances in the broader radio interferometry community, a mm-VLBI data simulator has been developed as described first in \citet{Blecher_2016} and has been expanded upon in this thesis.
%link to design objectives
We find that the realisation of our design objectives, laid out in \ref{sec:des_obj}, is apparent from the diversity of simulations shown in our results. 
%data formats
The pipeline uses the {\sc Measurement Set} format, in line with ALMA and future VLBI data formats. The focus has been placed on simulating realistic data given an arbitrary theoretical sky model. 
%experimental setup
We have placed focus has been placed on EHT observations, however, the pipeline is completely general with respect to observation configuration and allow any source structure in the form of {\sc fits} format. 
%time variability
Time variability in all domains (source, array, ISM, troposphere) is implemented. We highlight this point as we view the development of calibration and imaging routines which deal appropriately with source variability an essential challenge for observations of Sgr~A*. Distinguising complex gain variations, i.e. $\bm G$-Jones terms from short intrinsic variability will depend on the SNR obtained. The inclusion of ALMA in the array will be pivotal in this regard, however software advances can also add further utility.
%corruptions & formalism
To this end, the simulator includes signal corruptions in the interstellar medium (ISM), troposphere and instrumentation. Examples of typical corruptions have been demonstrated, which show that each corruption can significantly affect the inferred scientific parameters. A wide range of signal propagation effects can be implemented using the Measurement Equation formalism, with tropospheric scattering and antenna pointing errors given as illustrative examples. This formalism can also easily include bandpass imperfections and polarisation leakage.
%SPECIFIES, VALUES etc
%ISM
The ISM scattering implementation \textsc{ScatterBrane}, based on \citet*{Johnson_2015a}, has been incorporated into the pipeline. We have discussed how ISM substructure and variability can be difficult to disentangle from the instrinsic source structure especially if the source has spatial variability and samples different regions of the phase screen. The magnitude of the refractive substructure will also depend on the size of the emission region which at 1.3mm is sensitive to optical depth effects due to synchrotron self-absorption. If possible, observations of Sgr~A* should ideally be spaced apart by $r_{\rm ref}/v \sim$ a week  in order to sample indepedent realisations of the scattering screen. 


%src variability
Furthermore we have consider separating source variability into the following interlinked modes : a static flaring region, magnetic field/polarisation dynamics and primarly Stokes I spatial variability. Each of the different modes may require different approaches to calibration and parameter extraction. It might be necessary to split the observation into several sub-intervals which are calibrate separately although the determination of the sub-intervals is an open question. The utility of multi-wavelength data could assist with this. 


%trop im
Where we simulated images of a point sources with residual calibration errors, we find rapidly increasing flux attenuation from 1\% at 1\% turbulence to 36\% at 3\% turbulence. We also find offsets of a few microarcseconds. Interestingly the remnants of the PSF in the image are also distorted which can cause them to be difficult to identify as artefacts.


%antenna pointing
We have simulated the effects of various antenna pointing error models and conclude that stations with narrow beams like the LMT.  
Furthermore, we have briefly discussed the effects of the turbulent atmosphere on antenna pointing. This `anomalous' pointing error is potentially a serious calibration difficulty for sites without radiometers.
%


Applications for which the current version of the pipeline is well suited for include testing calibration and imaging routines in total intensity. One example which we have discussed is fringe-fitting in the presence of a variable troposphere with time-variable source.  
The creation of a close interface between sophisticated theoretical and interferometric mm-VLBI simulations will enhance the scientific opportunities possible with the EHT. 

%future
If the development of {\sc meqsilhouette} is continued, future capabilities would include handling a full Stokes source input and propagation through the pipeline as well as the simulation of polarisation leakage.  For easy access to interferometric simulations throughout the mm-VLBI community, a standard version of {\sc meqsilhouette} could be run through an online interface.  


We note that the foundation of the simulator has been build but more collaboration with the various EHT working groups is necessary for its evolution and the realisation of its utility which requires reliable data format converters.











