\Chapter{Conclusion}
%Leave till the end

In light of the science objectives of mm-VLBI observations and software advances in the broader radio interferometry community, a mm-VLBI data simulator has been developed. An important feature is that this simulation pipeline is performed using the {\sc Measurement Set} format, in line with ALMA and future VLBI data formats. The focus has been placed on simulating realistic data given an arbitrary theoretical sky model. To this end, the simulator includes signal corruptions in the interstellar medium (ISM), troposphere and instrumentation. Examples of typical corruptions have been demonstrated, which show that each corruption can significantly affect the inferred scientific parameters. Particular focus has been placed on EHT observations, however, the pipeline is completely general with respect to observation configuration and source structure. Time variability in all domains (source, array, ISM, troposphere) is implemented.  Future releases of \textsc{MeqSilhouette} will include polarisation dependent corruptions. The creation of a close interface between sophisticated theoretical and interferometric mm-VLBI simulations will enhance the scientific opportunities possible with the EHT.