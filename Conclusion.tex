\chapter{Conclusion}
\section{Summary}
%write with introduction.. possibly flesh out below
``In light of the science objectives of mm-VLBI observations and software advances in the broader radio interferometry community, a mm-VLBI data simulator has been developed. An important feature is that this simulation pipeline is performed using the {\sc Measurement Set} format, in line with ALMA and future VLBI data formats. The focus has been placed on simulating realistic data given an arbitrary theoretical sky model. To this end, the simulator includes signal corruptions in the interstellar medium (ISM), troposphere and instrumentation. Examples of typical corruptions have been demonstrated, which show that each corruption can significantly affect the inferred scientific parameters. Particular focus has been placed on EHT observations, however, the pipeline is completely general with respect to observation configuration and source structure. Time variability in all domains (source, array, ISM, troposphere) is implemented.  Future releases of \textsc{MeqSilhouette} will include polarisation dependent corruptions. The creation of a close interface between sophisticated theoretical and interferometric mm-VLBI simulations will enhance the scientific opportunities possible with the EHT.

%rewrite with references to proper sections
In section~\ref{sec:sim} we have described the layout of \textsc{MeqSilhouette} synthetic data simulation framework. A wide range of signal propagation effects can be implemented using the Measurement Equation formalism, with tropospheric scattering and antenna pointing errors given as illustrative examples. The framework is sufficiently general and flexible so that time variability in all relevant domains (source, array, ISM, troposphere) can be incorporated. The run time for a typical simulation with a realistic instrumental setup is on the order of minutes.  Implementation of polarisation effects is intended in the next version. 
''\\
%other sources of errors.. st 2
Whilst there are many additional potential sources of error (e.g. clock errors, bandpass, polarisation leakage, phasing errors, quantisation, correlator model etc.). The point of this investigation has been to demonstrate a mm-VLBI framework that enables more sophisticated interferometric simulations. As such, we have focused on capabilties not present in other mm-VLBI simulations and represent different Jones Matrix implmentations (i.e. the troposphere, ISM and antenna pointing errors). These also represent amongst the most challenging signal corruptions to implement. The MeqSilhouette framework, rooted in the Measurement Equation formalism, enables any arbitary error to be incorporated as a Jones Matrix (e.g. correlator model error is a simple scalar matrix. Our intention here is to demonstrate some of the key features of MeqSilhouette and its potential to provide realistic mm-VLBI simulations for systematic studies.

%link to design objectives
Apparent in these results is the realisation of the design objectives laid out in \ref{sec:des_obj}.

The ISM scattering implementation \textsc{ScatterBrane}, based on \citet*{Johnson_2015a}, has been incorporated into the pipeline.


\section{Suggested applications}

\subsubsection{Connecting theory to data}
Significant progress has been made in the theoretical and numerical modeling of the inner accretion flow and jet launch regions near a supermassive black hole event horizon
\citep[e.g.][]{Zanna_2007,Etienne_2010,Dexter_2013,Moscibrodzka_2014, McKinney_2014}. As the sensitivity of the EHT stands to dramatically increase, these theoretical efforts must be complemented by advances in interferometric simulations. With \textsc{MeqSilhouette}, we now have the ability to couple these with sophisticated interferometric and signal propagation simulations.  Moreover, detailed interferometric simulations will enable us to quantify systematic effects on the black hole and/or accretion flow parameter estimation.

\subsubsection{A public online interface}
%st 2
Table~\ref{tab:parameters} shows the set of parameters needed to run a standard {\sc meqsilhouette} simulation. This moderate number of parameters ($24+6N_{\rm stations}$) can be quickly chosen or selected from a list, especially if most of the defaults are preset and unlikely to change. This speaks to the possibility of a GUI interface which would provide the user with the capability to run standard simulations without having to delve into code. The capability to run such simulations would be useful to both theorists and observers in the broader AGN/SMBH/mm-VLBI community. For this reason we experimented with such an interface at the Leiden 2015 mm-VLBI workshop\footnote{http://www.astron.nl/other/workshop/mm-VLBI2015}, where we trialed an online interface for an early version of {\sc meqsilhouette}. We ran a short tutorial , which was well received by the researchers present. We are, however, yet to convert the latest version of the pipeline \citep{Blecher_2016} into such an interface, but hopefully this implementation will be made availible in the future when matters relating to public/propriety status are cleared up.



\section{Suggested improvements}

\subsubsection{Opacity and atmospheric brightness temperature fluctuations}

Run multiple realisations of atm to link the phase fluctuations to opacity and brightness temperature fluctuations possibly by fluctuating the input climate parameters., but this might be unfeasible.

\subsubsection{Full Stokes}

Later versions of {\sc MeqSilhouette} will enable the full Stokes cubes as input. This should not entail much work as our chosen data formats (MS, {\sc fits}) and the simulator in {\sc MeqTrees} already support full stokes logic. Signal propagation through the ISM and troposphere as well as antenna based complex gains errors are polarisation independent. The work would be primarly involve altering the existing scripts to dealing with the extra dimension in the {\sc fits} files i.e. book keeping. 
polarisation leakage in the RIME.

\subsubsection{GPU acceleration}





