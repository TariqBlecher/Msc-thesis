\chapter{Conclusion} %synthesis not summary..distill key and best ideas: 

%intro
In light of the science objectives of mm-VLBI observations and software advances in the broader radio interferometry community, a mm-VLBI data simulator has been developed \citep{Blecher_2016}.

%link to design objectives
Apparent in the results present is the realisation of the design objectives laid out in \ref{sec:des_obj}.

%data formats
An important feature is that this simulation pipeline is performed using the {\sc Measurement Set} format, in line with ALMA and future VLBI data formats. The focus has been placed on simulating realistic data given an arbitrary theoretical sky model. 


%experimental setup
Particular focus has been placed on EHT observations, however, the pipeline is completely general with respect to observation configuration and source structure. 
%time variability
Time variability in all domains (source, array, ISM, troposphere) is implemented. We highlight this point as we view the development of interferometric tools which deal appropriately with intrinsic source variability an essential challenge which for observations of Sgr~A*. 
%corruptions & formalism
To this end, the simulator includes signal corruptions in the interstellar medium (ISM), troposphere and instrumentation. Examples of typical corruptions have been demonstrated, which show that each corruption can significantly affect the inferred scientific parameters.

A wide range of signal propagation effects can be implemented using the Measurement Equation formalism, with tropospheric scattering and antenna pointing errors given as illustrative examples. 

%ISM
The ISM scattering implementation \textsc{ScatterBrane}, based on \citet*{Johnson_2015a}, has been incorporated into the pipeline.
%

%future
Future [improvements] of \textsc{meqsilhouette} will include polarisation dependent corruptions. 

fringe fitting variable troposphere with time variable soure
%closing

The creation of a close interface between sophisticated theoretical and interferometric mm-VLBI simulations will enhance the scientific opportunities possible with the EHT.''


more collaboration and working with EHT observing groups to ran their actual data processing algorithms on. this needs a good data format converter. 
%end on a positive note!



