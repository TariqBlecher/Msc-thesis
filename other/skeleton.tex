\documentclass{article} 
\usepackage{graphicx}
\usepackage{amsmath}



\begin{document}
\tableofcontents
\section{Introduction}

very brief overview and historical examples of how pushing to higher angular resolution (across the EM spectrum) has provided new discoveries and fundamental physical insights. E.g. compact radio cores and variability in quasars (before anyone knew what they were), superluminal motion, proto-planetary rings in HL Tau with ALMA; pick an HST result, etc. You can make a nice mosaic of 3 or 4 famous results. Maybe nice to start with an old figure of the first time Cygnus A was resolved? Or Jansky's map showing emission from the Galactic Centre? 

point is to introduce the overall benefits of high resolution and end with the comment that VLBI achieves best angular resolutions, which cues in your short intro into VLBI below.


\subsection{Very Long Baseline Interferometry}

this doesn't have to be too detailed, just a very brief historical context perhaps, what defines a VLBI array; examples of arrays and operating wavelengths, brightness temperature sensitivities, typical physics probed and angular resolution resolutions. Absolute best is with mm-VLBI with pushes the angular resolution limits, cue in EHT and the scientific opportunities....

basic schematic or figure (just cut from a textbook/paper) on VLBI

State here or start of next para that mm-VLBI includes a wide range of bands/arrays, however, you are going to focus on the EHT. The simulator can do any mm-VLBI array or frequency, however, the EHT focuses the science and given the immense challenge, the technical toolkit to be created will enable a wide range of scientific endeavours with this. 

\subsection{The Event Horizon Telescope}

brief intro

Remo's image of EHT station locations. maybe add small snapshots of some of the dishes to show how chunky they are. 


\subsection{Scientific objectives of the EHT}


$M_\odot$

\subsubsection{Gravity in the strong-field regime (?)}
Gravitational physics : measuring the shape and size of the bh shadow as a probe of strong gravity and bh spacetime. Constrain the no-hair theorem via measurement of quadrapole moment.\\
~\\
{\bf FIG. Plot of shapes and sizes of the bh shadow from the predictions of different theories of gravity}\\
{\it you may want to start with just an image of a shadow, before you even get into the different shapes? You're masters examiner isn't necessarily as clued up as your paper referee (or vice versa)}

~\\
\subsubsection{AGN accretion and jet-launch physics} 
Probing ordered magnetic fields, quiescent structure and variability on event horizon scales in both accretion disk and inner jet. Distinguish between the different Jet and Disk models for each bh. Deterimine spin.\\
~\\
\textbf{Fig: 2/3 panels of simulated images of disk and jet models of Sgr A*/M87}\\
\textbf{Fig: 2/3 panels of simulated polarimetric images of Sgr A*/M87 showing ordered magnetic fields}
{\it you may want to couple the above images with a diagram of what we think AGN look like: i.e. the twisted magnetic fields, bipolar jets emanating from spinning black hole. I would start with this in this subsection}


\subsection{Challenges and obstacles in mm-VLBI observations}

Briefly introduce signal corruptions, variability, ..etc, how these represent calibration and interpretation challenges.

{\it Make the key point that you dive into the detail in Chapter 2. }


\subsection{Robust scientific inference from mm-VLBI data}

{\it check out my section in the BHC review doc}
Briefly discuss imaging, it's difficulties..\\
~\\
Estimating the `macro'-parameters of Sgr A*, spin, orientation, position angle through a Bayesian parameter estimation analysis with closure quantities\\
~\\
\textbf{Fig: A Broderick 2016 posterior probability distribution (?)}\\


\subsection{A realistic mm-VLBI simulator}
Why simulate : Test calibration, imaging, parameter estimation algorithms. Build into calibration and analysis. Optimise observations.




\section{Theoretical Background}{\bf [Review theory underpinning the different components of the simulator ]}


\subsection{Radio Interferometry}
Briefly introduce interferometry. Necessity of interferometry to obtain adequate resolution. 
~\\
Define Fourier transform relationship between image - visibilities. \\
~\\
Static sky assumption and source variability
\subsubsection{Measurement Equation}
Signal propagation and Jones matricies.
	
\subsubsection{mm-VLBI observables and data products}

Visibility amplitudes, closure quantities, polarisation ratios, and images. 

\subsection{Signal corruptions}

\subsubsection{Introduction to scattering}
Similiar to paper
{\bf Fig. An illustration of basic scattering in the strong and weak regimes}

\subsubsection{Interstellar medium}

observations - history of scattering towards GC - blurring\\
                         {\bf Fig. Plot showing angular size of source as a function of frequency}

                          - recent evidence of refractive substructure and non-zero closure phases\\
                         {\bf Fig. Gwinn 2014 image of refractive scattering? or something at 1.3mm like Ortiz 2016}
                         
Review theory and simulation based on Johnson (2015) in more detail than paper

\subsubsection{Earth's atmosphere}

Review basic signal propagation effects : opacity, sky emission, delays. Relate to mm-VLBI observables.

{\bf Fig. Opacity and delay as a function of frequency}\\

{\bf Fig. Observations of phase variability fitting structure function analysis, Carilli and Holdaway 1997}\\


\subsubsection{Pointing error}
\subsubsection{Polarisation}

yes, i would include this and just give an overview of leakage, Faraday rotation,  and different feeds problem. say that MeqS not yet full Stokes, but will be soon and these corruptions will be included. 


\subsection{Bayesian parameter estimation (to do?) - yep, keep this here }


\section{Implementation}

{\bf Fig. MeqSilhouette flow chart}
\subsection{Simulation}

\subsection{Rodrigues}
{\bf Fig. Rodrigues screenshot}

\subsection{Parameter estimation (to do?)}



\section{Results and analysis}

\subsection{Canonical simulations}

Typical simulations of troposphere, ISM, pointing etc.\\

{\it It would be ideal for these to be what goes into standard challenges - kill several birds with one stone}

{\bf Fig. ISM variability: closure phases and amplitudes over 4 days}\\

{\bf Fig. Opacities and sky brightness temperatures for 3 sites}\\

{\bf Fig. Total and turbulent delays for ALMA and SMA towards Sgr A*}\\

{\bf Fig. Images with varying tropospheric corruptions}\\

{\bf Fig. LMT Pointing errors with three different models}\\

{\bf Tropospheric induced closure phase errors}
 We can discuss this, but I think it warrants a mention. The errors are fairly small ($\sigma_{\rm cp} \sim$ few degrees ) under 100\% turbulence and physical or not it's an interesting consequence of how the simulator has been implemented.

\subsection{Parameter estimation (to do? - yep)}

I'm thinking about a bunch of different applications, one of which is the binary SMBH candidate I showed you a few months back. We can develop this more fully when you visit Grahamstown. I think it would be a nicer 2nd half of the thesis than implementing the full Stokes version. 


\section{Discussion}

\section{Conclusions}

\section{Future work}
MeqS plan (as per Lindy's EHT doc) for the examiner's interest and context. 

\end{document}