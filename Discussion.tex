\chapter{chap:discussion}

NON-CLOSING ERRORS
Non-closing errors due to incoherent averaging - "no conjugates for triples" - possibly comes down to how one defines SNR. In the definition used in the literature we followed, they assumed only gaussian noise, which was not the case..A better definition would be to look at the distribution of a number of samples 


Why no seeing? - seeing results when the decoherence is considered as a function of baseline length, longer baseline are more decoherent and their visibilities are downweighted. For the EHT as different stations would experience independent different turbulent intensities, the baseline length will not be correlated with the magnitude of the seeing. But this will still distort the image, leading to uncharacteristic feature extraction....ummm taken the blurring sentence out of the paper on roger's instruction



\section{Possible future improvements to the simulator}

\subsection{Full Stokes}

Later versions of {\sc MeqSilhouette} will enable the full Stokes cubes as input. This should not entail much work as our chosen data formats (MS, {\sc fits}) and the simulator in {\sc MeqTrees} already support full stokes logic. Signal propagation through the ISM and troposphere as well as antenna based complex gains errors are polarisation independent. The work would be primarly involve altering the existing scripts to dealing with the extra dimension in the {\sc fits} files i.e. book keeping. 
polarisation leakage in the RIME.

\subsection{Opacity and atmospheric brightness temperature fluctuations}

Run multiple realisations of atm to link the phase fluctuations to opacity and brightness temperature fluctuations possibly by fluctuating the input climate parameters., but this might be unfeasible.
