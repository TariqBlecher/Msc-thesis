\chapter{Software documentation}
%st 1
The following appendix serves as documentation for the installation and use of {\sc meqsilhouette}.

\section{Installation}
%st 1

The {\sc meqsilhouette} repository is currently private, however if made public it is maintained here\footnote{https://github.com/ratt-ru/MeqSilhouette}. Currently, the simulator is running on {\sc Ubuntu 14.04} and has not been tested on other operating systems. Software requirements which are being maintained elsewhere and are publicly available: 
\begin{itemize}
 \item {\sc simms}\footnote{https://github.com/radio-astro/simms}
 \item {\sc MeqTrees}\footnote{https://ska-sa.github.io/meqtrees/} 
 \item {\sc Scatterbrane}\footnote{http://krosenfeld.github.io/scatterbrane}
\end{itemize}

Note that {\sc simms} in turn requires {\sc casa}, where we have had success with version 4.2.2. Several different routes for the installation of {\sc MeqTrees} are available\footnote{https://github.com/ska-sa/meqtrees/wiki/Installation}. Note that as {\sc aatm} is not currently being maintained or easily available, we include it within the {\sc meqsilhouette} repository.


Once the {\sc meqsilhouette} repository has been cloned, either add the framework module to the PYTHONPATH directly or install using pip,
\begin{verbatim}
$ cd MeqSilhouette/framework/; sudo pip install .
\end{verbatim}

\section{Usage}

%st 1
We will first discuss the simple case of running a simulation with the canonical pre-written driver script. Following this, we will discuss how to write one's own driver script.

\subsection{Running a standard simulation}
%st1

We will focus on the standard `azishe' pipeline, the central concepts are captured in Fig.~\ref{flow}, however antenna pointing errors are not included and the sky model needs to be in {\sc fits} format. 

To run in the MeqSilhouette repository we pass the driver script and parameter dictionary to {\sc Python},
\begin{verbatim}
$python driver/azishe.py input/parameters.json
\end{verbatim}


The content of the parameter dictionary is shown, slightly altered, in table~\ref{tab:parameters}. The parameters in the dictionary can be altered by editing the dictionary directly. Note that if the content of antenna table is changed, so should the content of station information file which contains station $SEFD$ and weather information. The path of the folder containing the sky model is also located in the parameter dictionary, for multiple fits files, i.e. a time-variable source, the order of their implementation is the same as if their names with sorted with numpy.sort, where each fits files is observed for approximately the same amount of time. If the ISM-scattering is turned on, the extra logistics of moving and creating fits files and folders are automatically handled. The primary log is set by `v.LOG' variable, intialised in the driver script. The output will be found in the `output' directory, within a subdirectory whose name will be the same as the fitsfolder in the parameter dictionary. 

\subsection{Writing a driver script}
 
A number of different pipelines are available in the driver scripts folder, which should be provide `worked examples' when writing a novel driver script. The most important steps are:
\begin{enumerate}
 \item setup parameter dictionary as well as sub-dictionaries
 \item create measurement set
 \item initialise SimCoordinator class
 \item Use SimCoordinator to generate visibilities and apply signal corruptions
\end{enumerate}

Often it is useful to perform iterations of subset of steps in the pipeline which is straightforward with a \emph{for} loop and several simple functions or clearing and copying measurement sets. Additional instrumental signal corruptions should ideally be implemented within the TropMS class. It is important that the pipeline respects the causality of signal transmission path and/or the commutativity of Jones-matricies.

















